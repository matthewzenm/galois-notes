\section{分裂域}
给定一个域$K$及$K$上的多项式$p(x)\in K[x]$, 我们希望找到一个扩域$L/K$使得$p(x)$在$L$上``有所有的根''.
给``有根''这一点以严格的定义, 我们便得到了\textbf{分裂域}的概念:
\begin{defn}
    设$K$是域, $p(x)\in K[x]$, 如果扩域$L/K$使得$p(x)$在$L$上可以分解为一次因式的乘积 (简称为\textbf{分裂})
    \[p(x)=(x-\alpha_1)\cdots(x-\alpha_n)\]
    且$L=K(\alpha_1,\cdots,\alpha_n)$, 那么称$L$是$p(x)$在$K$上的\textbf{分裂域}.
\end{defn}

在证明分裂域的存在性与唯一性之前, 我们先给出一些分裂域的例子.
\begin{eg}给定底域$K$, 讨论多项式$p(x)\in K[x]$.
    \begin{enumerate}
        \item $p(x)=x-a_0$, $a_0\in K$, 那么分裂域就是$K$.
        \item $p(x)=x^2-a_1x+a_0$, $a_1,a_0\in K$, 且$p(x)$不可约.
        那么$L=K[x]/\langle p(x)\rangle$包含了$p(x)$的一个根$\alpha$, 而事实上, $L$也包含了另一个根$a_1-\alpha$.
        所以$L$是$p(x)$的一个分裂域.
        \item 取$K=\mathbb{Q}$及$p(x)=x^3-2$. $L=\mathbb{Q}(\sqrt[3]{2})=\mathbb{Q}[x]/\langle x^3-2\rangle$包含了$\sqrt[3]{2}$, 但不包含$x^3-2$的复根, 此时$x^3-2=(x-\sqrt[3]{2})(x^2+\sqrt[3]{2}x+\sqrt[3]{4})$.
        于是取$M=L[y]/\langle y^2+\sqrt[3]{2}y+\sqrt[3]{4}$, 则$M$是一个分裂域, 并且有$[M:k]=6$.
        \item $p(x)=8x^3+4x^2-4x-1$.
    \end{enumerate}
\end{eg}

我们现在证明分裂域的存在性.
\begin{proof}
    给定域$K$及$p=p_1p_2\cdots p_m\in K[x]$, 其中$p_i\ (i=1,\cdots,m)$均不可约.
    我们对$\deg{p}$用归纳法.
    当$\deg{p}=1$时, $K$本身就是$p(x)$的分裂域.
    假设对$\deg{p}=n-1$成立.
    对$\deg{p}=n$, 考虑域$K_1=K[x]/\langle p_1(x)\rangle$, 那么$p$在$K_1$上至少有一个根$\alpha$, $p$在$K_1$上可以分解为
    \[p(x)=(x-\alpha)p_a(x)\]
    对$p_a(x)$用归纳假设, 存在扩域$L/K_1$使得$p_a(x)$分裂为一次因式的乘积, 从而在扩域$L/K$上$p(x)$分裂为一次因式的乘积.
    由归纳原理得证.
\end{proof}

我们着手证明分裂域的同构唯一性.
我们把这个命题加强为\textit{分裂域作为域扩张是同构唯一的}, 即对域$K$及分裂域$L,L'$, 有如下的图表交换
\[\begin{tikzcd}
    L\ar[rr, "\sim"] & & L'\\
    & K \ar[ul] \ar[ur] &
\end{tikzcd}\]

\begin{thm}[分裂域的同构唯一性]
    设$K$是域, $p(x)\in K[x]$, 域$K'$与$K$同构, 且$p(x)$在同构映射下的像为$p'(x)$.
    设$L,L'$分别是$p(x),p'(x)$的分裂域, 那么存在同构$L\to L'$使得以下图表交换
    \[\begin{tikzcd}
        L\ar[r, "\sim"] & L'\\
        K\ar[r, "\sim"] \ar[u] & K' \ar[u]
    \end{tikzcd}\]
\end{thm}
\begin{proof}
    设$i:K\xrightarrow{\sim}K'$是同构, 我们也用$i$表示延拓到$K[x]\to K'[x]$的同构.
    依然对$p(x)$的次数用归纳法.
    $\deg{p}=1$时, $K=L,K'=L'$, 命题显然成立.
    假设命题对$\deg{p}=n-1$成立, 那么对$p$的某个不可约因子$p_1$, 有
    \[K(\alpha)=\frac{K[x]}{\langle p(x)\rangle}\simeq\frac{K'[x]}{\langle i(p(x))\rangle}=K(\alpha')\]
    从而可以得到交换图
    \[\begin{tikzcd}
        K(\alpha)\ar[r,"\sim"] & K'(\alpha)\\
        K\ar[u]\ar[r,"\sim"] & K'\ar[u]
    \end{tikzcd}\]
    而在$K(\alpha),K'(\alpha')$上$p(x),p'(x)$分别分解为一次因式与一个$n-1$次多项式的乘积, 从而按归纳假设, 可以得到两个$n-1$次多项式的分裂域的同构
    \[\begin{tikzcd}
        L\ar[r, "\sim"] & L'\\
        K(\alpha)\ar[r, "\sim"] \ar[u] & K'(\alpha') \ar[u]
    \end{tikzcd}\]
    从而有大图表
    \[\begin{tikzcd}
        L\ar[r, "\sim"] & L'\\
        K(\alpha)\ar[r, "\sim"] \ar[u] & K'(\alpha') \ar[u]\\
        K\ar[r,"\sim"] \ar[u] & K'\ar[u]
    \end{tikzcd}\]
    交换, 即得到所欲证命题.
\end{proof}

\begin{rem}
    需要注意到, 两个分裂域之间的同构不一定是唯一的.
    例如分别使用$i,j$表示虚数单位, $x^2+1$在$\mathbb{R}$上的两个分裂域
    \[\begin{tikzcd}
        \mathbb{C}=\mathbb{R}(i)\ar[rr, "h"] & & \mathbb{R}(j)\\
        & \mathbb{R}\ar[ul] \ar[ur] &
    \end{tikzcd}\]
    其中$h:\mathbb{R}(i)\to\mathbb{R}(j)$可以取为$i\mapsto j$与$i\mapsto -j$, 得到两个同构.
\end{rem}