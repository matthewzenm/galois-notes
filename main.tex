\documentclass[11pt]{article}

\usepackage{ctex}
\usepackage[paper=b5paper]{geometry}
\usepackage{amsmath}
\usepackage{amssymb}
\usepackage{amsthm}
\usepackage[toc]{multitoc}
\usepackage{biblatex}
\usepackage[colorlinks]{hyperref}

\addbibresource{biblio.bib}
\defbibheading{bibliography}[本笔记用到的参考文献]{\subsection*{#1}}

% Theorem environments
\theoremstyle{definition}
\newtheorem{defn}{定义}[section]
% \newtheorem{sym}[defn]{记号}
\newtheorem{eg}[defn]{例}
\theoremstyle{plain}
\newtheorem{thm}[defn]{定理}
\newtheorem{lem}[defn]{引理}
\newtheorem{col}[defn]{推论}
\newtheorem{prop}[defn]{命题}
% \newtheorem*{pro}{问题}
\theoremstyle{remark}
\newtheorem{rem}[defn]{评注}
% \newtheorem{ex}[defn]{习题}

\title{域与Galois理论笔记}
\author{魔法少女Alkali}
\date{最后编译: \today}

\begin{document}
\maketitle

\section{前言}
这份笔记是笔者学习Fields奖得主Richard Borcherds所讲授的网课\textit{Galois Theory} (链接:~\href{https://www.youtube.com/watch?v=ccc4EYeytYo}{YouTube}~或~\href{https://www.bilibili.com/video/BV1Uy4y1i7sh}{bilibili}) 时记录的笔记.
原始的笔记是英文的, 但笔者思考之后决定还是使用中文整理出最终的笔记.

这份笔记不是网课的逐字稿, Borcherds教授所讲的内容中有些部分没有被记录下来 (例如正十七边形的具体构造), 也有一些教授略过的部分被详细地补充 (例如任意集合的分裂域的同构延拓定理).
更多地, 这份笔记被整理成了笔者心目中适合自己和他人阅读的模样.
因此, 这份笔记便不可避免地带有了笔者的个人色彩, 从而许多地方的讲法与证明并不一定是最好的.
更为致命的是, 本份笔记是作者为备考中科院2023年``代数与数论''暑期学校而突击整理的笔记 (虽然最终没能在考前整理出来), 因此错误应当俯拾即是, 所以还盼望读者指正.

联系我可以通过\href{mailto:matthewzenm@icloud.com}{我的邮箱}.
\nocite{*}
\printbibliography

\end{document}