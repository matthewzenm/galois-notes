\documentclass[]{article}

\usepackage{ctex}
\usepackage[paper=b5paper]{geometry}
\usepackage{amsmath}
\usepackage{amssymb}
\usepackage{amsthm}
\usepackage[toc]{multitoc}
\usepackage{biblatex}
\usepackage[shortlabels]{enumitem}
\setlist{nosep}
\usepackage{tikz-cd}
\usepackage[colorlinks]{hyperref}

\addbibresource{biblio.bib}
\defbibheading{bibliography}[本笔记用到的参考文献]{\subsection*{#1}}

% Theorem environments
\theoremstyle{definition}
\newtheorem{defn}{定义}[section]
\newtheorem{sym}[defn]{记号}
\newtheorem{eg}[defn]{例}
\theoremstyle{plain}
\newtheorem{thm}[defn]{定理}
\newtheorem{lem}[defn]{引理}
\newtheorem{col}[defn]{推论}
\newtheorem{prop}[defn]{命题}
\newtheorem*{pro}{问题}
\theoremstyle{remark}
\newtheorem{rem}[defn]{评注}
% \newtheorem{ex}[defn]{习题}

\title{域与Galois理论笔记}
\author{魔法少女Alkali}
\date{最后编译: \today}

\begin{document}
\maketitle

\setcounter{section}{-1}
\section{前言}
这份笔记是笔者学习Fields奖得主Richard Borcherds所讲授的网课\textit{Galois Theory} (链接:~\href{https://www.youtube.com/watch?v=ccc4EYeytYo}{YouTube}~或~\href{https://www.bilibili.com/video/BV1Uy4y1i7sh}{bilibili}) 时记录的笔记.
原始的笔记是英文的, 但笔者思考之后决定还是使用中文整理出最终的笔记.

这份笔记不是网课的逐字稿, Borcherds教授所讲的内容中有些部分没有被记录下来 (例如正十七边形的具体构造), 也有一些教授略过的部分被详细地补充 (例如任意集合的分裂域的同构延拓定理).
更多地, 这份笔记被整理成了笔者心目中适合自己和他人阅读的模样.
因此, 这份笔记便不可避免地带有了笔者的个人色彩, 从而许多地方的讲法与证明并不一定是最好的.
更为致命的是, 本份笔记是作者为备考中科院2023年``代数与数论''暑期学校而突击整理的笔记 (虽然应该没有办法在考前整理出来), 因此错误应当俯拾即是, 所以还盼望读者指正.

联系我可以通过\href{mailto:matthewzenm@icloud.com}{我的邮箱}.
\nocite{*}
\printbibliography

\section{域扩张}

我们先给出域扩张的定义.
\begin{defn}
    设$K,L$是域, 且满足$K\subset L$, 那么称$L$是$K$的一个\textbf{扩域}, 记作$L/K$.
\end{defn}

\begin{eg}
    我们最熟悉的扩域的例子是$\mathbb{Q}\subset\mathbb{R}\subset\mathbb{C}$.
\end{eg}

\begin{defn}设有域扩张$L/K$.
    \begin{enumerate}
        \item 定义扩张的\textbf{度数}为$[L:K]=\dim_KL$, 当$[L:K]<\infty$时, 称$L/K$为\textbf{有限扩张};
        \item 设$\alpha\in L$, 如果$\alpha$是某个多项式$p(x)\in K[x]$的根, 那么称$\alpha$在$K$上是\textbf{代数}的, 否则称为是\textbf{超越}的;
        \item 设$\alpha$是$K$上的代数元, 设$p(x)\in K[x]$是$\alpha$的极小多项式, 即以$\alpha$为根的次数最低的多项式, 那么定义$\alpha$的度数$\deg\alpha=\deg p(x)$.
    \end{enumerate}
\end{defn}

\begin{eg}我们给出一些域扩张的例子.
    \begin{enumerate}
        \item 对$\mathbb{Q}\subset\mathbb{R}\subset\mathbb{C}$, 有$[\mathbb{C}:\mathbb{R}]=2,[\mathbb{R}:\mathbb{Q}]=\infty$.
        (具体而言, $[\mathbb{R}:\mathbb{Q}]=2^{\aleph_0}$)
        \item 取$K=\mathbb{Q}$, 那么$\alpha=\sqrt[5]{2}$是代数的, $\pi, e\in\mathbb{R}$是超越的.
        \item 对$\mathbb{Q}\subset\mathbb{Q}(x)$, 即$\mathbb{Q}$上的有理函数域作为$\mathbb{Q}$的扩域, $x$在$\mathbb{Q}$上是超越的.
        \item $\alpha=\cos(2\pi/7)$是代数的.
        注意到对$\zeta=e^{2\pi/7}$, 有$\alpha=(\zeta+\zeta^{-1})/2$.
        而
        \begin{align*}
            1+\zeta+\cdots+\zeta^6=0&\implies \zeta^{-3}+\zeta^{-2}+\cdots+1+\cdots+\zeta^3=0\\
            &\implies (2\alpha)^3+(2\alpha)^2-2(2\alpha)-1=0\\
            &\iff 8\alpha^3+4\alpha-4\alpha-1=0
        \end{align*}
        所以$\alpha$是$\mathbb{Q}$上的代数元.
    \end{enumerate}
\end{eg}

\begin{sym}
    对$L/K$及集合$S\subset L$, 我们记$K(S)$为包含$S$中所有元素的最小的扩域, 并称为\textbf{由$S$生成的扩域}.
    特别地, 当$S=\{\alpha_1,\cdots,\alpha_n\}$时, 我们记$K(S)=K(\alpha_1,\cdots,\alpha_n)$.
\end{sym}

考虑由单个代数元$\alpha$生成的扩域, 我们有如下的引理
\begin{lem}
    设$L/K$, $\alpha\in L$是$K$上的代数元, 有极小多项式$m(x)\in K[x]$, 那么$K(\alpha)\simeq K[x]/\langle m(x)\rangle$, 其中$\langle m(x)\rangle$是$p(x)$生成的理想.
\end{lem}
\begin{proof}
    定义同态
    \begin{align*}
        \varphi:K[x]&\to K(\alpha)\\
        p(x)&\mapsto p(\alpha)
    \end{align*}
    考虑核$\ker\varphi$, 显然$\ker\varphi\neq K[x]$, 且极小多项式$m(x)\in\ker\varphi$.
    由于$K[x]$是主理想整环, $\ker\varphi$单生成, 且生成元整除$m(x)$.
    但容易证明$m(x)$是不可约多项式, 结合$\ker\varphi\neq K[x]$可知生成元与$m(x)$相伴, 从而$\ker\varphi=\langle m(x)\rangle$.
    由第一同构定理即知
    \[K(\alpha)\simeq \frac{K[x]}{\langle m(x)\rangle}\qedhere\]
\end{proof}

关于有限扩张与代数扩张, 有如下的结论
\begin{thm}
    设有域扩张$M/K$, $\alpha\in M$是$K$上的代数元当且仅当$\alpha$包含在$K$的一个有限扩张中.
\end{thm}
\begin{proof}
    一方面, 假设$\alpha$是代数元, 那么$\alpha\in K(\alpha)$.
    设$\deg\alpha=n$, 那么$1,\alpha,\cdots,\alpha^{n-1}$是$K(\alpha)$的一组基, $K(\alpha)/K$是有限扩张.
    另一方面, 假设$\alpha$包含在$K$的有限扩张中, 不妨设$[M:K]=n<\infty$.
    那么$1,\alpha,\cdots,\alpha^{n-1},\alpha^n$一定线性相关, 从而$\alpha$是一个多项式的根, 是一个代数元.
\end{proof}

\begin{thm}[望远镜公式]\label{telescope}
    设$K\subset L\subset M$均为有限扩张, 那么有$[M:K]=[M:L][L:K]$
\end{thm}
\begin{proof}
    设$x_1,\cdots,x_m$是$L/K$的一组基, $y_1,\cdots,y_n$是$M/L$的一组基.
    我们考虑$\{x_iy_j\}_{(i,j)\subset [m]\times[n]}$\footnote{$[m]=\{1,\cdots,m\}$, 组合数学中的常用记号.}.
    首先对$a_{ij}\in K$及指标$(i,j)\in R\times S\subset [m]\times[n]$有
    \begin{align*}
        &\sum_{(i,j)\in R\times S}a_{ij}(x_iy_j)=0\\
        \implies&\sum_{j\in S}a_{ij}y_j=0,\ \forall i\in R\\
        \implies&a_{ij}=0,\ \forall (i,j)\in R\times S
    \end{align*}
    所以$x_iy_j$线性无关.
    其次, 显然$M$中的每个元素可以表示为$x_iy_j$的$K$--线性组合, 所以$\{x_iy_j\}_{(i,j)\subset [m]\times[n]}$是$M/L$的一组基.
    从而命题得证.
\end{proof}

通过望远镜公式, 我们可以证明
\begin{thm}
    设$\alpha,\beta$是$K$上的代数元, 那么$\alpha\pm\beta,\alpha\beta,\alpha/\beta(\beta\neq 0)$均为$K$上的代数元.
\end{thm}
\begin{proof}
    考虑扩张链$K\subset K(\alpha)\subset K(\alpha,\beta)$, 两个扩张均为代数扩张, 所以都是有限扩张.
    由定理~\ref{telescope}, $K(\alpha,\beta)/K$是代数扩张.
    而$\alpha\pm\beta,\alpha\beta,\alpha/\beta$均包含在$K(\alpha,\beta)$中, 所以都是代数元.
\end{proof}

\begin{thm}\label{algcoef}
    设$\alpha$是一个由$K$上代数元系数构成的多项式的根, 那么$\alpha$是代数的.
\end{thm}
\begin{proof}
    设
    \[\alpha^n+a_{n-1}\alpha^{n-1}+\cdots+a_0=0\]
    且$a_{n-1},\cdots,a_0$均为$K$上代数元.
    考虑域扩张链
    \begin{align*}
        K&\subset K(a_0)\\
        &\subset K(a_0,a_1)\\
        &\cdots\\
        &\subset K(a_0,\cdots,a_{n-1})\\
        &\subset K(a_0,\cdots,a_{n-1},\alpha)
    \end{align*}
    前$n$步扩张每一步都是添加一个代数元$a_i$, 所以都是有限的, 因此$K$上的扩域$K(a_0,\cdots,a_{n-1})$是有限的.
    而由假设, $\alpha$在$K(a_0,\cdots,a_{n-1})$上代数, 所以最后一步扩张也是有限的.
    因此扩张$K(a_0,\cdots,a_{n-1},\alpha)/K$是有限的, 从而$\alpha$是$K$上代数元.
\end{proof}

而关于超越元, 我们已知$e,\pi$在$\mathbb{Q}$上是超越的, 但是有如下的公开问题
\begin{pro}
    $e+\pi,e\pi$在有理数域上超越吗?
\end{pro}

不过我们可以有这样的结论
\begin{prop}
    $e+\pi,e\pi$至多有一个是代数的.
\end{prop}
\begin{proof}
    否则$e+\pi,e\pi$都是代数的, 由定理~\ref{algcoef}~可知方程
    \begin{equation}
        x^2-(e+\pi)x+e\pi=0\label{e and pi}
    \end{equation}
    的根是代数的.
    但方程~\eqref{e and pi}~的根是$e$和$\pi$, 这与我们已知的$e$与$\pi$的超越性矛盾.
\end{proof}

\section{分裂域}
给定一个域$K$及$K$上的多项式$p(x)\in K[x]$, 我们希望找到一个扩域$L/K$使得$p(x)$在$L$上``有所有的根''.
给``有根''这一点以严格的定义, 我们便得到了\textbf{分裂域}的概念:
\begin{defn}
    设$K$是域, $p(x)\in K[x]$, 如果扩域$L/K$使得$p(x)$在$L$上可以分解为一次因式的乘积 (简称为\textbf{分裂})
    \[p(x)=(x-\alpha_1)\cdots(x-\alpha_n)\]
    且$L=K(\alpha_1,\cdots,\alpha_n)$, 那么称$L$是$p(x)$在$K$上的\textbf{分裂域}.
\end{defn}

在证明分裂域的存在性与唯一性之前, 我们先给出一些分裂域的例子.
\begin{eg}给定底域$K$, 讨论多项式$p(x)\in K[x]$.
    \begin{enumerate}
        \item $p(x)=x-a_0$, $a_0\in K$, 那么分裂域就是$K$.
        \item $p(x)=x^2-a_1x+a_0$, $a_1,a_0\in K$, 且$p(x)$不可约.
        那么$L=K[x]/\langle p(x)\rangle$包含了$p(x)$的一个根$\alpha$, 而事实上, $L$也包含了另一个根$a_1-\alpha$.
        所以$L$是$p(x)$的一个分裂域.
        \item 取$K=\mathbb{Q}$及$p(x)=x^3-2$. $L=\mathbb{Q}(\sqrt[3]{2})=\mathbb{Q}[x]/\langle x^3-2\rangle$包含了$\sqrt[3]{2}$, 但不包含$x^3-2$的复根, 此时$x^3-2=(x-\sqrt[3]{2})(x^2+\sqrt[3]{2}x+\sqrt[3]{4})$.
        于是取$M=L[y]/\langle y^2+\sqrt[3]{2}y+\sqrt[3]{4}$, 则$M$是一个分裂域, 并且有$[M:k]=6$.
        \item $p(x)=8x^3+4x^2-4x-1$.
    \end{enumerate}
\end{eg}

我们现在证明分裂域的存在性.
\begin{proof}
    给定域$K$及$p=p_1p_2\cdots p_m\in K[x]$, 其中$p_i\ (i=1,\cdots,m)$均不可约.
    我们对$\deg{p}$用归纳法.
    当$\deg{p}=1$时, $K$本身就是$p(x)$的分裂域.
    假设对$\deg{p}=n-1$成立.
    对$\deg{p}=n$, 考虑域$K_1=K[x]/\langle p_1(x)\rangle$, 那么$p$在$K_1$上至少有一个根$\alpha$, $p$在$K_1$上可以分解为
    \[p(x)=(x-\alpha)p_a(x)\]
    对$p_a(x)$用归纳假设, 存在扩域$L/K_1$使得$p_a(x)$分裂为一次因式的乘积, 从而在扩域$L/K$上$p(x)$分裂为一次因式的乘积.
    由归纳原理得证.
\end{proof}

我们着手证明分裂域的同构唯一性.
我们把这个命题加强为\textit{分裂域作为域扩张是同构唯一的}, 即对域$K$及分裂域$L,L'$, 有如下的图表交换
\[\begin{tikzcd}
    L\ar[rr, "\sim"] & & L'\\
    & K \ar[ul] \ar[ur] &
\end{tikzcd}\]

\begin{thm}[分裂域的同构唯一性]
    设$K$是域, $p(x)\in K[x]$, 域$K'$与$K$同构, 且$p(x)$在同构映射下的像为$p'(x)$.
    设$L,L'$分别是$p(x),p'(x)$的分裂域, 那么存在同构$L\to L'$使得以下图表交换
    \[\begin{tikzcd}
        L\ar[r, "\sim"] & L'\\
        K\ar[r, "\sim"] \ar[u] & K' \ar[u]
    \end{tikzcd}\]
\end{thm}
\begin{proof}
    设$i:K\xrightarrow{\sim}K'$是同构, 我们也用$i$表示延拓到$K[x]\to K'[x]$的同构.
    依然对$p(x)$的次数用归纳法.
    $\deg{p}=1$时, $K=L,K'=L'$, 命题显然成立.
    假设命题对$\deg{p}=n-1$成立, 那么对$p$的某个不可约因子$p_1$, 有
    \[K(\alpha)=\frac{K[x]}{\langle p(x)\rangle}\simeq\frac{K'[x]}{\langle i(p(x))\rangle}=K(\alpha')\]
    从而可以得到交换图
    \[\begin{tikzcd}
        K(\alpha)\ar[r,"\sim"] & K'(\alpha)\\
        K\ar[u]\ar[r,"\sim"] & K'\ar[u]
    \end{tikzcd}\]
    而在$K(\alpha),K'(\alpha')$上$p(x),p'(x)$分别分解为一次因式与一个$n-1$次多项式的乘积, 从而按归纳假设, 可以得到两个$n-1$次多项式的分裂域的同构
    \[\begin{tikzcd}
        L\ar[r, "\sim"] & L'\\
        K(\alpha)\ar[r, "\sim"] \ar[u] & K'(\alpha') \ar[u]
    \end{tikzcd}\]
    从而有大图表
    \[\begin{tikzcd}
        L\ar[r, "\sim"] & L'\\
        K(\alpha)\ar[r, "\sim"] \ar[u] & K'(\alpha') \ar[u]\\
        K\ar[r,"\sim"] \ar[u] & K'\ar[u]
    \end{tikzcd}\]
    交换, 即得到所欲证命题.
\end{proof}

\begin{rem}
    需要注意到, 两个分裂域之间的同构不一定是唯一的.
    例如分别使用$i,j$表示虚数单位, $x^2+1$在$\mathbb{R}$上的两个分裂域
    \[\begin{tikzcd}
        \mathbb{C}=\mathbb{R}(i)\ar[rr, "h"] & & \mathbb{R}(j)\\
        & \mathbb{R}\ar[ul] \ar[ur] &
    \end{tikzcd}\]
    其中$h:\mathbb{R}(i)\to\mathbb{R}(j)$可以取为$i\mapsto j$与$i\mapsto -j$, 得到两个同构.
\end{rem}
\section{代数闭包}
\begin{defn}
    设$K$是一个域, 如果扩域$\overline{K}/K$满足
    \begin{enumerate}[(1)]
        \item $K[x]$中的任意多项式在$\overline{K}$中均分裂;
        \item $\overline{K}$由$K[x]$中多项式的根生成,
    \end{enumerate}
    那么称$\overline{K}$是$K$的\textbf{代数闭包}.
\end{defn}

我们给出代数闭包的构造.
\begin{thm}
    任意域$K$均存在代数闭包.
\end{thm}
\begin{lem}
    设$L/K$是代数扩张, 那么有$|L|\leq\max\{|K|,|\mathbb{N}|\}$.
\end{lem}
\begin{proof}
    我们有分解
    \[L=\bigcup_{n\geq 1}\{\alpha\in L:\ \deg\alpha=n\}\]
    而对每个$\{\alpha\in L:\ \deg\alpha=n\}$中的元素$\alpha$, $\alpha$与另外至多$n-1$个元素与$K$中$n$个系数决定的首一多项式对应, 从而有
    \[\{\alpha\in L:\ \deg\alpha=n\}\subset [n]\times K^n\]
    对无限的$K$而言, $|[n]\times K^n|=|K|$, 从而
    \begin{align*}
        |L|&=\left|\bigcup_{n\geq 1}\{\alpha\in L:\ \deg\alpha=n\}\right|\\
        &\leq|\mathbb{N}\times K|\\
        &=|K|
    \end{align*}
    对有限的$F$而言, $|[n]\times K^n|=n|K|^n\leq|\mathbb{N}|$, 此时
    \begin{align*}
        |L|&=\left|\bigcup_{n\geq 1}\{\alpha\in L:\ \deg\alpha=n\}\right|\\
        &\leq|\mathbb{N}\times\mathbb{N}|\\
        &=|\mathbb{N}|
    \end{align*}
    综上, 可以得到
    \[|L|\leq\max\{|K|,|\mathbb{N}|\}\qedhere\]
\end{proof}
\begin{proof}[代数闭包存在性的证明]
    设$A$是$K$上所有代数扩域的集合.
    取$S$满足$F\subset S$且$|S|>\max\{|K|,|\mathbb{N}|\}$, 那么由引理, $K$的代数扩张均包含在$S$中, 从而$A\subset\mathcal{P}(S)$是一个集合.
    使用包含关系作为偏序, 那么注意到对任意一条链$c:(\{K_i\},\subset)$, 易见$\bigcup_{i\geq 1}K_i$是$c$的一个上界.
    因此由Zorn引理, $A$中存在极大元$M$.
    断言在$M$中任意$p(x)\in K[x]$分裂.
    否则假设存在一个$p(x)$在$M$上不能分解为一次因式的乘积, 那么设$p(x)$在$M$上具有分裂域$E$, $E/M,M/K$都是代数扩张, 从而$E/K$是代数扩张 (推论~\ref{alg of alg}), $E\in A$.
    然而$M\subsetneq E$, 这与$M$在$A$中的极大性矛盾.
    因此$M$中任意$p(x)\in K[x]$分裂, 取$M$的由$K[x]$中所有多项式的根生成的子域$\overline{K}$即得到$K$的代数闭包.
    (证明中用到的集合论结论可以参考~\parencite[附录2第2, 3节]{Lang})
\end{proof}

关于代数闭包, 有一个密切相关的概念是代数闭域:
\begin{defn}
    域$L$被称为是\textbf{代数闭域}, 如果$L[x]$中的任意多项式都在$L$中有根.
\end{defn}
\begin{prop}
    域$K$的代数闭包$\overline{K}$是代数闭域.
\end{prop}
\begin{proof}
    设$p(x)=x^n+a_{n-1}x^{n-1}+\cdots+a_0,\ a_i\in\overline{K}$.
    由于$\overline{K}$由$K[x]$中多项式的根生成, 因此$a_i$均为$K$上的代数元.
    对$p(x)$在某个根$\alpha$, 考虑扩张链
    \[K\subset K(a_0,\cdots,a_{n-1})\subset K(a_0,\cdots,a_{n-1},\alpha)\]
    容易发现两个扩张都是有限的, 所以$\alpha$也是$K$上的代数元, 从而在$\overline{K}$内.
    因此$\overline{K}$是代数闭域.
\end{proof}

我们接下来讨论一种弱于代数闭的性质.
我们希望找到一个扩域$L/K$, 使得$L$在开根号下封闭.
\begin{proof}[构造]
    想法是不断地添加平方根.
    取$K_0=K$, $K_1$为$K_0$上所有形如$x^2-a,\ a\in K_0$的多项式的分裂域 (它包含在$K_0$的一个代数闭包中, 所以存在).
    递归地定义$K_{n+1}$为$K_n$上所有形如$x^2-b,\ b\in K_n$的多项式的分裂域.
    取$L=\bigcup_{n\in\mathbb{N}}K_n$, 那么容易验证$L$是一个域;
    同时对任意$\alpha\in L$, 存在某个$K_i$使得$\alpha\in K_i$, 那么$\alpha$的平方根按定义在$K_{i+1}\subset L$中.
    因此$L$关于开根号封闭.
\end{proof}

接下来我们给出一些代数闭包的例子.
\begin{eg}
    我们最熟悉的代数闭包莫过于$\mathbb{R}$的代数闭包$\mathbb{C}$.
    这个结论被称为代数基本定理.
    % 在之后我们会在第~\ref{fta}~节利用Galois理论证明代数基本定理, 但是比较简单的方法是利用复分析中的Liouville定理或者卷绕数.
    在之后我们会利用Galois理论证明代数基本定理, 但是比较简单的方法是利用复分析中的Liouville定理或者卷绕数.
    对这些证明, 可以参考~\parencite[命题8.13]{Lvovski}.
\end{eg}

其他的一些``自然''的代数闭包的例子有
\begin{eg}
    \begin{enumerate}
        \item $\mathbb{C}/\mathbb{R}$;
        \item 有理数的代数闭包$\overline{\mathbb{Q}}\subset\mathbb{C}$, 称为代数数.
        \item 考虑形式Laurent级数$\mathbb{C}[[x]][x^{-1}]$, 它的代数闭包被称为Puiseux级数, 即
        \[\bigcup_{n\geq 1}\mathbb{C}[[x^{1/n}]][x^{-1/n}]\]
        证明参考~\parencite[命题II.8]{Serre}.
    \end{enumerate}
\end{eg}

现在我们证明代数闭包的同构唯一性.
我们证明一个更强的命题
\begin{thm}[同构延拓定理]\label{iso ext thm}
    设$K$是一个域, $S\subset K[x]$是一族多项式, $K'$与$K$同构且$S$在同构映射下的像为$S'$.
    设$E,E'$分别是$S,S'$的分裂域, 那么存在同构$S\to S'$使得下图交换
    \[\begin{tikzcd}
        S\ar[r, "\sim"] & S'\\
        K\ar[u] \ar[r, "\sim"] & K\ar[u]
    \end{tikzcd}\]
\end{thm}
\begin{proof}
    设$A$是由子域与嵌入$(F,\tau)$构成的集合, 其中$K\subset F\subset E$且使得下图交换
    \[\begin{tikzcd}
        E' & & \\
        K\ar[u, "\sigma"] \ar[r] & F \ar[ul, "\tau"'] \ar[r] & E
    \end{tikzcd}\]
    我们在$A$上定义偏序$(F,\tau)\prec(F',\tau')$当且仅当$F\subset F'$且$\tau'|_F=\tau$.
    对任意一条链$\{(F_i,\tau_i)\}$, 取$F=\bigcup_{i\geq 0}F_i$, $\tau:F\to E'$满足$\tau|_{F_i}=\tau_i$.
    那么容易验证$(F,\tau)$是这条链的一个上界.
    由Zorn引理, $A$中存在一个极大元$(M,\tilde{\sigma})$.
    断言$M=E$. 否则的话存在一个$S$中的多项式$p(x)$在$M$上不分裂, 那么对$p(x)$的一个根$\alpha$, 可以按下图延拓得到$\tilde{\sigma}':M(\alpha)\to E'$
    \[\begin{tikzcd}
         & E'\\
        M(\alpha)\ar[r, "\tilde{\sigma}_\alpha"] \ar[ur, dashed, "\tilde{\sigma}'"] & \tilde{\sigma}(M)(\alpha') \ar[u]\\
        M \ar[r, "\tilde{\sigma}"] \ar[u] & \tilde{\sigma}(M) \ar[u]
    \end{tikzcd}\]
    这与$M$的极大性矛盾, 所以$M=E$.
    注意到$E$包含了$S$中所有多项式的根, 并被$\tilde{\sigma}$一一地映到$E'$中.
    而$E'$是包含$S'$中所有多项式的根的最小的域, 所以一定有$\tilde{\sigma}(E)=E'$.
    因此命题得证.
\end{proof}

\begin{rem}
    我们指出代数闭包间的同构也不是唯一的.
    并且我们也无法``自然''地找出两个代数闭包之间的同构, 也就是说$\overline{K}$的\textbf{绝对Galois群}是没有单位元的.
    这种情形与拓扑空间$X$中的道路的同伦类$\pi_1(X)$相似: 我们可以定义道路的同伦类之间的乘法, 但是无法自然地找到单位元.
    在这种情形下, 我们会把$\Aut(\overline{K}/K)$及$\pi_1(X)$称为\textbf{群胚}.
    在范畴论中, 群胚被定义为所有态射都是同构的范畴.
\end{rem}

\end{document}