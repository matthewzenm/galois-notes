\documentclass[]{article}

\usepackage{ctex}
\usepackage[paper=b5paper]{geometry}
\usepackage{amsmath}
\usepackage{amssymb}
\usepackage{amsthm}
\usepackage[toc]{multitoc}
\usepackage{biblatex}
\usepackage[shortlabels]{enumitem}
\setlist{nosep}
\usepackage{tikz-cd}
\usepackage[colorlinks]{hyperref}

\addbibresource{biblio.bib}
\defbibheading{bibliography}[本笔记用到的参考文献]{\subsection*{#1}}

% Theorem environments
\theoremstyle{definition}
\newtheorem{defn}{定义}[section]
\newtheorem{sym}[defn]{记号}
\newtheorem{eg}[defn]{例}
\theoremstyle{plain}
\newtheorem{thm}[defn]{定理}
\newtheorem{lem}[defn]{引理}
\newtheorem{col}[defn]{推论}
\newtheorem{prop}[defn]{命题}
\newtheorem*{pro}{问题}
\theoremstyle{remark}
\newtheorem{rem}[defn]{评注}
% \newtheorem{ex}[defn]{习题}

\DeclareMathOperator{\Aut}{Aut}
\DeclareMathOperator{\Gal}{Gal}
\DeclareMathOperator{\GF}{GF}
\DeclareMathOperator{\ch}{char}

\title{域与Galois理论笔记}
\author{魔法少女Alkali}
\date{最后编译: \today}

\begin{document}
\maketitle

\setcounter{section}{-1}
\section{前言}
这份笔记是笔者学习Fields奖得主Richard Borcherds所讲授的网课\textit{Galois Theory} (链接:~\href{https://www.youtube.com/watch?v=ccc4EYeytYo}{YouTube}~或~\href{https://www.bilibili.com/video/BV1Uy4y1i7sh}{bilibili}) 时记录的笔记.
原始的笔记是英文的, 但笔者思考之后决定还是使用中文整理出最终的笔记.

这份笔记不是网课的逐字稿, Borcherds教授所讲的内容中有些部分没有被记录下来 (例如正十七边形的具体构造), 也有一些教授略过的部分被详细地补充 (例如任意集合的分裂域的同构延拓定理).
更多地, 这份笔记被整理成了笔者心目中适合自己和他人阅读的模样.
因此, 这份笔记便不可避免地带有了笔者的个人色彩, 从而许多地方的讲法与证明并不一定是最好的.
更为致命的是, 本份笔记是作者为备考中科院2023年``代数与数论''暑期学校而突击整理的笔记 (虽然应该没有办法在考前整理出来), 因此错误应当俯拾即是, 所以还盼望读者指正.

联系我可以通过\href{mailto:matthewzenm@icloud.com}{我的邮箱}.
\nocite{*}
\printbibliography

\section{域扩张}
\section{分裂域}
给定一个域$K$及$K$上的多项式$p(x)\in K[x]$, 我们希望找到一个扩域$L/K$使得$p(x)$在$L$上``有所有的根''.
给``有根''这一点以严格的定义, 我们便得到了\textbf{分裂域}的概念:
\begin{defn}
    设$K$是域, $p(x)\in K[x]$, 如果扩域$L/K$使得$p(x)$在$L$上可以分解为一次因式的乘积 (简称为\textbf{分裂})
    \[p(x)=(x-\alpha_1)\cdots(x-\alpha_n)\]
    且$L=K(\alpha_1,\cdots,\alpha_n)$, 那么称$L$是$p(x)$在$K$上的\textbf{分裂域}.
\end{defn}

在证明分裂域的存在性与唯一性之前, 我们先给出一些分裂域的例子.
\begin{eg}给定底域$K$, 讨论多项式$p(x)\in K[x]$.
    \begin{enumerate}
        \item $p(x)=x-a_0$, $a_0\in K$, 那么分裂域就是$K$.
        \item $p(x)=x^2-a_1x+a_0$, $a_1,a_0\in K$, 且$p(x)$不可约.
        那么$L=K[x]/\langle p(x)\rangle$包含了$p(x)$的一个根$\alpha$, 而事实上, $L$也包含了另一个根$a_1-\alpha$.
        所以$L$是$p(x)$的一个分裂域.
        \item 取$K=\mathbb{Q}$及$p(x)=x^3-2$. $L=\mathbb{Q}(\sqrt[3]{2})=\mathbb{Q}[x]/\langle x^3-2\rangle$包含了$\sqrt[3]{2}$, 但不包含$x^3-2$的复根, 此时$x^3-2=(x-\sqrt[3]{2})(x^2+\sqrt[3]{2}x+\sqrt[3]{4})$.
        于是取$M=L[y]/\langle y^2+\sqrt[3]{2}y+\sqrt[3]{4}$, 则$M$是一个分裂域, 并且有$[M:k]=6$.
        \item $p(x)=8x^3+4x^2-4x-1$.
    \end{enumerate}
\end{eg}

我们现在证明分裂域的存在性.
\begin{proof}
    给定域$K$及$p=p_1p_2\cdots p_m\in K[x]$, 其中$p_i\ (i=1,\cdots,m)$均不可约.
    我们对$\deg{p}$用归纳法.
    当$\deg{p}=1$时, $K$本身就是$p(x)$的分裂域.
    假设对$\deg{p}=n-1$成立.
    对$\deg{p}=n$, 考虑域$K_1=K[x]/\langle p_1(x)\rangle$, 那么$p$在$K_1$上至少有一个根$\alpha$, $p$在$K_1$上可以分解为
    \[p(x)=(x-\alpha)p_a(x)\]
    对$p_a(x)$用归纳假设, 存在扩域$L/K_1$使得$p_a(x)$分裂为一次因式的乘积, 从而在扩域$L/K$上$p(x)$分裂为一次因式的乘积.
    由归纳原理得证.
\end{proof}

我们着手证明分裂域的同构唯一性.
我们把这个命题加强为\textit{分裂域作为域扩张是同构唯一的}, 即对域$K$及分裂域$L,L'$, 有如下的图表交换
\[\begin{tikzcd}
    L\ar[rr, "\sim"] & & L'\\
    & K \ar[ul] \ar[ur] &
\end{tikzcd}\]

\begin{thm}[分裂域的同构唯一性]
    设$K$是域, $p(x)\in K[x]$, 域$K'$与$K$同构, 且$p(x)$在同构映射下的像为$p'(x)$.
    设$L,L'$分别是$p(x),p'(x)$的分裂域, 那么存在同构$L\to L'$使得以下图表交换
    \[\begin{tikzcd}
        L\ar[r, "\sim"] & L'\\
        K\ar[r, "\sim"] \ar[u] & K' \ar[u]
    \end{tikzcd}\]
\end{thm}
\begin{proof}
    设$i:K\xrightarrow{\sim}K'$是同构, 我们也用$i$表示延拓到$K[x]\to K'[x]$的同构.
    依然对$p(x)$的次数用归纳法.
    $\deg{p}=1$时, $K=L,K'=L'$, 命题显然成立.
    假设命题对$\deg{p}=n-1$成立, 那么对$p$的某个不可约因子$p_1$, 有
    \[K(\alpha)=\frac{K[x]}{\langle p(x)\rangle}\simeq\frac{K'[x]}{\langle i(p(x))\rangle}=K(\alpha')\]
    从而可以得到交换图
    \[\begin{tikzcd}
        K(\alpha)\ar[r,"\sim"] & K'(\alpha)\\
        K\ar[u]\ar[r,"\sim"] & K'\ar[u]
    \end{tikzcd}\]
    而在$K(\alpha),K'(\alpha')$上$p(x),p'(x)$分别分解为一次因式与一个$n-1$次多项式的乘积, 从而按归纳假设, 可以得到两个$n-1$次多项式的分裂域的同构
    \[\begin{tikzcd}
        L\ar[r, "\sim"] & L'\\
        K(\alpha)\ar[r, "\sim"] \ar[u] & K'(\alpha') \ar[u]
    \end{tikzcd}\]
    从而有大图表
    \[\begin{tikzcd}
        L\ar[r, "\sim"] & L'\\
        K(\alpha)\ar[r, "\sim"] \ar[u] & K'(\alpha') \ar[u]\\
        K\ar[r,"\sim"] \ar[u] & K'\ar[u]
    \end{tikzcd}\]
    交换, 即得到所欲证命题.
\end{proof}

\begin{rem}
    需要注意到, 两个分裂域之间的同构不一定是唯一的.
    例如分别使用$i,j$表示虚数单位, $x^2+1$在$\mathbb{R}$上的两个分裂域
    \[\begin{tikzcd}
        \mathbb{C}=\mathbb{R}(i)\ar[rr, "h"] & & \mathbb{R}(j)\\
        & \mathbb{R}\ar[ul] \ar[ur] &
    \end{tikzcd}\]
    其中$h:\mathbb{R}(i)\to\mathbb{R}(j)$可以取为$i\mapsto j$与$i\mapsto -j$, 得到两个同构.
\end{rem}
\section{代数闭包}
\begin{defn}
    设$K$是一个域, 如果扩域$\overline{K}/K$满足
    \begin{enumerate}[(1)]
        \item $K[x]$中的任意多项式在$\overline{K}$中均分裂;
        \item $\overline{K}$由$K[x]$中多项式的根生成,
    \end{enumerate}
    那么称$\overline{K}$是$K$的\textbf{代数闭包}.
\end{defn}

我们给出代数闭包的构造.
\begin{thm}
    任意域$K$均存在代数闭包.
\end{thm}
\begin{proof}
    使用良序定理 (\parencite[附录2, 定理4.1]{Lang}) 在$K[x]$上赋予良序$(K[x],\prec)$.
    对任意一条链$c:p_1\prec p_2\prec\cdots$, 我们取$K_0=K$, $K_1$为$p_1$在$K_0$上的分裂域, $K_2$为$p_2$在$K_1$上的分裂域, 得到域扩张链
    \[F_c:K_0\subset K_1\subset K_2\subset\cdots\]
    考虑$K^c=\bigcup_{i\in\mathbb{N}}K_i$, 那么$K^c$是$F_c$的上界, 并且$K^c$恰好包含了$c$中所有多项式的根.
    取$X$是每一条链$F_c$与$K^c$的集合, 那么由Zorn引理 (\parencite[附录2第2节]{Lang}), $X$中存在极大元$\overline{K}$.
    一方面, 任意$p(x)\in K[x]$都在$\overline{K}$中分裂, 否则具有$p(x)$的一个根$\alpha$使得$\overline{K}\subsetneq\overline{K}(\alpha)$, 与$\overline{K}$的极大性矛盾.
    另一方面, $X$中没有添加$K[x]$中多项式的根以外的元素, 所以$\overline{K}$恰好由$K[x]$中多项式的根生成.
    因此$\overline{K}$就是$K$的代数闭包.
\end{proof}

关于代数闭包, 有一个密切相关的概念是代数闭域:
\begin{defn}
    域$L$被称为是\textbf{代数闭域}, 如果$L[x]$中的任意多项式都在$L$中有根.
\end{defn}
\begin{prop}
    域$K$的代数闭包$\overline{K}$是代数闭域.
\end{prop}
\section{有限域}
回忆整数到一个域$K$有一个自然的同态
\begin{align*}
    \lambda:\mathbb{Z}&\to K\\
    m&\mapsto m\cdot 1
\end{align*}
如果$\ker\lambda=\mathbb{Z}$, 那么称$K$的\textbf{特征}为$0$;
如果$\ker\lambda=\langle n\rangle$, 那么称$K$的特征为$n>0$.
记$K$的特征为$\ch{K}$.
容易证明, 当$\ch{K}=0$时, $K$一定包含$\mathbb{Q}$作为子域;
当$\ch{K}>0$时, $K$的特征一定是素数 (设为$p$), 且包含$\mathbb{Z}/p\mathbb{Z}=\mathbb{F}_p$作为子域.
我们将$\mathbb{Q}$与$\mathbb{F}_p$ ($p$素数)称为\textbf{素域}.

假设$F$是有限域, 那么$F$一定有正的特征$p>0$.
那么此时素域$\mathbb{F}_p\subset F$, $F$是$\mathbb{F}_p$上的向量空间.
如果$\dim_{\mathbb{F}_p}F=n$, 那么每个坐标分量有$p$种取法, 则$|F|=p^n$.
因此我们得到
\begin{prop}
    有限域$F$的阶为$p^n$, 其中$p=\ch{F}$是素数, $n=[F:\mathbb{F}_p]$.
\end{prop}

相同的论证我们可以得到
\begin{prop}\label{finite subfield}
    有限域$\mathbb{F}_{p^n}\subset\mathbb{F}_{p^m}$当且仅当$n|m$.
\end{prop}

\begin{sym}
    对素数$p$, 在上下文意义明确时我们记它的一个方幂$p^n:=q$.
    对$q$阶有限域, 我们将其记为$\GF(q)=\GF(p^n)=\mathbb{F}_q=\mathbb{F}_{p^n}$.\footnote{似乎这需要先证明有限域是唯一的, 不过这件事情之后我们确实会做.}
\end{sym}

首先我们证明有限域的存在性.
\begin{thm}
    对素数$p$及$q=p^n$, 存在$q$阶有限域.
\end{thm}
\begin{proof}
    取$x^q-x$在$\mathbb{F}_p$上的一个分裂域$L$, 我们证明$L$恰好由$x^q-x$的所有根构成.
    我们先证明$x^q-x$的根构成一个域.
    对根$x,y$, 由$\ch L=p$可知$\binom{q}{k}=0,\ k=1,\cdots,q-1$, 从而
    \begin{align*}
        (x-y)^q&=x^q-y^q\quad(p=2\text{时}1=-1,\ \text{所以均写为减号})\\
        &=x-y
    \end{align*}
    所以$x-y$是$x^q-x$的一个根;
    而当$y\neq 0$时
    \begin{align*}
        \left(\frac{x}{y}\right)^q-\frac{x}{y}&=\frac{x^qy-xy^q}{y^{q+1}}\\
        &=\frac{xy-yx}{y^{q+1}}\\
        &=0
    \end{align*}
    所以$x/y$也是一个根.
    因此$x^q-x$的根在减法与除法下封闭, 构成一个域.
    由于分裂域由根生成, 所以$L$恰好由$x^q-x$的根构成.
    另一方面, 由于$(x^q-x)'=qx^{q-1}-1=-1$, 与$x^q-x$互素, 所以$x^q-x$没有重根.
    因此$|L|=\deg(x^q-x)=q$.
\end{proof}

然后我们证明有限域的唯一性.
\begin{thm}
    两个有限域同构当且仅当他们阶数相同.
\end{thm}
\begin{proof}
    设有限域$F$的阶数为$q$, 我们证明$F$一定是$x^q-x$的分裂域.
    这只需要证明对任意$a\in F$有$a^q=a$即可.
    $a=0$时这是平凡的.
    对$a\in F^*$, 由Lagrange定理, $a^{|F^*|}=1$, 即$a^{q-1}=1$, 从而$a^q=a$.
    因此$F$是$x^q-x$的分裂域, 在同构意义下是唯一的.
\end{proof}

对于给定的$q$, 我们希望问
\begin{pro}
    如何构造$q$阶有限域?
\end{pro}
回答很简单, 我们取一个$n$次不可约多项式$f(x)\in\mathbb{F}_p[x]$, 那么就有$\GF(q)=\mathbb{F}_p[x]/\langle f(x)\rangle$.
我们看一个例子.

\begin{eg}
    给定$p=2$.
    我们写一些低次数的不可约多项式:
    \begin{gather*}
        x,\ x+1,\\
        x^2+x+1,\\
        x^3+x+1,\ x^3+x^2+1,\\
        x^4+x+1,\ x^4+x^3+x^2+x+1,\ x^4+x^3+1
    \end{gather*}
    那么我们有
    \begin{enumerate}[(1)]
        \item 次数为$4=2^2$: $\GF(4)=\mathbb{F}_2[x]/\langle x^2+x+1\rangle$, 习惯上把$x$记为三次单位根$\omega$, 域中的元素为
        \[0,1,\omega,\omega+1\]
        \item 次数为$8=2^3$: $\GF(8)=\mathbb{F}_2[x]/\langle x^3+x+1\rangle$, 此时域中的元素为
        \[0,1,x,x+1,x^2,x^2+1,x^2+x,x^2+x+1\]
        同时也有$\GF(8)=\mathbb{F}_2[y]/\langle y^3+y^2+1\rangle$, 这两种构造下的域应当是同构的.
        事实上, 注意到$(y+1)^3+(y+1)+1=0$, 所以同构映射可以由$x=y+1$给出.
    \end{enumerate}
\end{eg}

通过以上这个例子, 我们可以看出确实没有``典范''的构造有限域的方法.

最后, 我们讨论求有限域上不可约多项式的个数的问题.
我们只在有限素域上考虑这个问题.

\begin{prop}\label{counting}
    设$F_d$(x)是$\mathbb{F}_p$上所有$d$次不可约多项式的乘积, 那么有
    \[x^{p^n}-x=\prod_{d|n}F_d(x)\]
\end{prop}
\begin{lem}
    $\mathbb{F}_p[x]$中的不可约多项式均没有重根.
\end{lem}
\begin{proof}
    设$f(x)\in\mathbb{F}_p[x]$不可约.
    如果$f'(x)\neq 0$, 那么$(f(x),f'(x))=1$, 从而$f(x)$没有重根.
    如果$f'(x)=0$, 那么$f(x)$一定具有形式 (不妨设首一)
    \begin{align*}
        f(x)&=x^{np}+a_{n-1}x^{(n-1)p}+\cdots+a_1x^p+a_0\\
        &=(x^n+a_{n-1}x^{n-1}+\cdots+a_0)^p
    \end{align*}
    与$f(x)$不可约矛盾.
    所以$f(x)$没有重根.
\end{proof}
\begin{proof}[命题~\ref{counting}~的证明]
    首先我们说明如果次数至少为$1$的多项式$f(x)|x^{p^n}-x$, 那么$f^2(x)\nmid x^{p^n}-x$.
    事实上如果有$x^{p^n}-x=f^2(x)g(x)$, 那么计算形式导数有
    \[-1=2f(x)f'(x)g(x)+f^2(x)g'(x)\]
    从而$f(x)|1$, 矛盾.
    其次我们说明$f(x)|x^{p^n}-x$当且仅当$d=\deg{f(x)}|n$.
    设$L$是$x^{p^n}-x$的分裂域, 即$\GF(p^n)$.
    对$f(x)$的一个根$\alpha$, 考虑$\mathbb{F}_p(\alpha)$.
    那么$[\mathbb{F}_p(\alpha):\mathbb{F}_p]=d$, 由命题~\ref{finite subfield}, $\mathbb{F}_p(\alpha)\subset L$当且仅当$d|n$, 即$x-\alpha|x^{p^n}-x$当且仅当$d|n$.
    因此$f(x)|x^{p^n}-x$时一定有$x-\alpha|x^{p^n}-x$, 从而$d|n$;
    $d|n$时$f(x)$ (在$L$上) 的所有根$\alpha,\beta,\cdots$满足$x-\alpha,x-\beta,\cdots$均整除$x^{p^n}-x$, 又因为$f(x)$没有重根, 这些一次因式两两互素, 有$f(x)|x^{p^n}-x$.
    综上可知命题成立.
\end{proof}

对命题~\ref{counting}~使用M\"{o}bius变换 (\parencite[第2章定理2]{NT}), 我们可以得到
\begin{thm}
    $\mathbb{F}_p[x]$上$n$次不可约多项式的个数为
    \[\frac{1}{n}\sum_{d|n}\mu\left(\frac{n}{d}\right)p^d\]
\end{thm}

\begin{col}
    对任意正整数$n$, $\mathbb{F}_p[x]$中存在$n$次不可约多项式.
\end{col}
\begin{proof}
    $n$次不可约多项式的个数为$n^{-1}(p^n\pm\cdots+p\mu(n))$, 括号中的式子被$p$恰整除 (即$p^2$不整除这个式子), 所以一定不是$0$.
\end{proof}
\section{正规扩张与可分扩张}
对于一个代数扩张$L/K$及$K$上的一个多项式$p(x)\in K[x]$, 假设$L$中包含了$p(x)$的一个根.
那么我们会问, $L$包含了$p(x)$的所有根吗?
看以下两个例子.
\begin{eg}
    取底域为$\mathbb{Q}$.
    \begin{enumerate}[(1)]
        \item $p(x)=x^3-2$在$\mathbb{Q}(\sqrt[3]{2})$上有一个根, 但是$\mathbb{Q}(\sqrt[3]{2})$不包含$x^3-2$的复根;
        \item $p(x)=8x^3+4x^2+4x-1$, 取$\alpha=\cos{2\pi/7}$, 那么$\alpha$是$p(x)$的一个根.
        而$p(x)$的另外两个根为$\cos{4\pi/7}=2a^2-1,\cos{6\pi/7}=2(2\alpha^2-1)^2-1$, 均在$\mathbb{Q}(\alpha)$中.
    \end{enumerate}
\end{eg}

以下我们给出关于代数扩张的三个等价命题, 每一个命题都可以用来描述\textbf{正规扩张}.

\begin{thm}\label{normal thmdef}
    设$K\subset L$是代数扩张, 那么以下三个命题等价:
    \begin{enumerate}[(1)]
        \item $K[x]$中任何在$L$上有根的多项式$p(x)$在$L[x]$中分裂;
        \item $L$是$K$上某一族多项式的分裂域;
        \item $\overline{K}$的所有固定$K$不动的自同构都将$L$映为$L$.
    \end{enumerate}
\end{thm}

\begin{defn}
    满足定理~\ref{normal thmdef}~中三个等价条件中任意一个的代数扩张称为\textbf{正规扩张}.
\end{defn}

\begin{proof}[定理~\ref{normal thmdef}~的证明]
    我们按照$(1)\implies(2)\implies(3)\implies(1)$的顺序证明它们等价.\\
    $(1)\implies(2)$: 对任意$\alpha\in L$, 由于$L/K$是代数扩张, 因此可以取$\alpha$在$K$上的极小多项式$p_\alpha(x)$, 那么$L$是
    \(S=\{p_\alpha\in K[x]:\ \alpha\in L\}\)
    的分裂域.\\
    $(2)\implies(3)$: 设$L$是$S\subset K[x]$的分裂域, $\sigma\in\Aut{\overline{K}}$固定$K$不动.
    那么$\sigma$也固定$S$中多项式的系数不动, 从而$S$中多项式的根也被固定不动.
    而$L$是$S$中多项式的根的分裂域, 所以也被固定不动.\\
    $(3)\implies(1)$: 设$p(x)\in K[x]$具有一个根$\alpha\in L$, 不妨设$p(x)$不可约.
    设$\beta\in\overline{K}$是$p(x)$的另一个根, 由于$p(x)$不可约, 所以存在固定$K$不动的同构$K(\alpha)\to K(\beta)$.
    按照同构延拓定理 (定理~\ref{iso ext thm}), 这个同构可以延拓为$\overline{K}$的自同构$\sigma:\overline{K}\to\overline{K}$
    \[\begin{tikzcd}
        K \ar[d, "="] \ar[r] & K(\alpha) \ar[d, "\sim"] \ar[r] & \overline{K} \ar[d, "\sigma"]\\
        K \ar[r] & K(\beta) \ar[r] & \overline{K}
    \end{tikzcd}\]
    那么按照假设, $\sigma$将$L$中的$\alpha$映成$L$中的$\beta$, 即$\beta\in L$.
    从而$p(x)$的根均在$L$中, $p(x)$在$L[x]$中分裂.
\end{proof}

\begin{eg}我们再看几个例子.
    \begin{enumerate}[(1)]
        \item 我们在前面知道了$\mathbb{Q}(\sqrt[3]{2})/\mathbb{Q}$不是正规的, 但$\mathbb{Q}(\sqrt[3]{2},\omega)/\mathbb{Q}$是正规的.
        \item 如果$K\subset L$且$[L:K]=2$, 那么$L/K$一定是正规的:
        $L$一定是单生成的, 否则存在三个$K$--线性无关的元素$1,\alpha,\beta$, 矛盾.
        那么设$L=K(\alpha)$, 由于$[L:k]=2$, $\alpha$的极小多项式是二次的, 不妨设为$x^2+bx+c$.
        从而极小多项式的另一个根为$-b-\alpha\in L$, 所以极小多项式在$L[x]$中分裂, $L/K$是正规的.
        \item 假设$L/K,M/L$都是正规扩张, 那么$M/K$是正规扩张吗?
        答案是否定的, 一个简单的例子是$\mathbb{Q}\subset\mathbb{Q}(\sqrt{2})\subset\mathbb{Q}(\sqrt[4]{2})$.
    \end{enumerate}
\end{eg}

接下来我们给出可分多项式, 可分元及可分扩张的定义.
\begin{defn}
    设$K$是域, $f(x)\in K[x]$, 如果$f(x)$在$\overline{K}$中没有重根, 那么称$f(x)$是一个\textbf{可分多项式}.
    如果$K\subset L$, $\alpha\in L$是一个可分多项式的根, 那么称$\alpha$是\textbf{可分元}.
    如果$L/K$中每个元素都是可分的, 那么称$L/K$是\textbf{可分扩张}.
\end{defn}

可分多项式的一个简单的判别法是
\begin{prop}
    $f(x)\in K[x]$是可分多项式当且仅当它的形式导数$f'(x)$与其互素.
\end{prop}
\begin{proof}
    设$\alpha$是$f(x)$的一个根, 那么$f(x)$ (在分裂域或者代数闭包上) 有分解
    \[f(x)=(x-\alpha)^{n_1}f_1(x),\ f_1(\alpha)\neq 0\]
    当$n_1\geq 2$时, 求导有
    \[f'(x)=n_1(x-\alpha)^{n_1-1}f_1(x)+(x-\alpha)^{n_1}f_1'(x)\]
    仍被$x-\alpha$整除, 从而$x-\alpha|(f,f')$, 两者不互素.
    当$n_1=1$时, 求导的结果为
    \[f'(x)=f_1(x)+(x-\alpha)^{n_1}f_1'(x)\]
    此时$f'(\alpha)=f_1(\alpha)\neq 0$, 从而$x-\alpha\nmid f'(x)$.
    那么当$f$可分时, $f$的任意根都不是$f'$的根, 从而$(f,f')=1$.
\end{proof}

\begin{eg}我们给出一些可分或不可分扩张的例子.
    \begin{enumerate}[(1)]
        \item 设$K\subset L$是代数扩张, $\ch{K}=0$, 那么$L$是可分扩张.
        事实上, 对任意$\alpha\in L$, 取$\alpha$在$K$上的极小多项式$f(x)$, 那么$f(x)$在$K$上是不可约的.
        从而对非零的$f'(x)$一定有$(f,f')=1$, 即$\alpha$的极小多项式可分, $\alpha$是可分元.
        \item 而当$\ch{K}>0$时, $K$的代数扩张不一定是可分的.
        因为此时$f'(x)$可以是$0$, 比如$f(x)=a_nx^{np}+a_{n-1}x^{(n-1)p}+\cdots+a_0$.
        具体地举一个例子, 设$k$是一个特征$p$的域, 取$K=k(t^p),L=k(t)$, 其中$t$是$k$上的超越元.
        那么$L/K$是有限扩张, 从而是代数的.
        但$t$不是可分元: $t$在$K$上具有极小多项式$x^p-t^p$, 但$x^p-t^p=(x-t)^p$只有一个根.
        \item 有限域的代数扩张都是可分的.
        事实上设$\alpha$是$\mathbb{F}_q$上的代数元, 设其是$n$次的, 那么$[\mathbb{F}_q(\alpha):\mathbb{F}_q]=n$.
        从而$\alpha$是多项式$f(x)=x^{q^n}-x$的根, 而$f'(x)=-1$, 与$f(x)$互素, 从而$f(x)$可分, 则$\alpha$可分.
    \end{enumerate}
\end{eg}

而对于不可分扩张, 最极端的情况是如下定义的纯不可分扩张
\begin{defn}
    设$K\subset L$, $\ch{K}=p$, 如果对任意$\alpha\in L$, $\alpha$都是$K$上某个多项式$x^{p^n}-a$的根, 那么称$L/K$是\textbf{纯不可分扩张}.
\end{defn}

对一般的代数扩张, 我们有如下命题
\begin{thm}
    设$L/K$是代数扩张, 那么$L/K$可以分解为$K\subset K^{\mathrm{sep}}\subset L$, 使得$K^{\mathrm{sep}}/K$是可分扩张, $L/K^{\mathrm{sep}}$是纯不可分扩张.
    $K^{\mathrm{sep}}$称为$K$的\textbf{可分闭包}.
\end{thm}
由于本笔记中只处理可分扩张, 所以这个命题的证明我们直接引用~\parencite[第V章, 命题6.6]{Lang}


% \section{代数基本定理}\label{fta}

我们知道代数基本定理有许多证明, 有使用拓扑方法的, 复分析方法的, 更简单的可以使用微积分的方法.
在这一节我们利用Galois理论给出代数基本定理的一个代数证明.

\end{document}